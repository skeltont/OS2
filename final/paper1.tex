

\subsection{Processes}
\label{sec:Processes}

\subsubsection{FreeBSD}
\label{sub:Process FreeBSD}

\par FreeBSD defines processes as a program in execution.
The process is initialized with it's own address space that contains mappings of its object code and all global variables.
The process is also allotted a set of kernel resources, such as its credentials, signal state and descriptor array (for I/O). \cite{bsd:1}
The processes' data structure must be stored in memory for the duration of it's execution.
This data structure, however, can be allocated to and freed from memory dynamically as it begins and terminates.

\par In FreeBSD, each process is assigned a unique identifier that is called its PID (Process Identifier) \cite{bsd:1}.
A running process will only track two PIDs- that of its parent and its own.
This is stored in the process structure, along with information like: signal state, process tracing information, and timers (both real and CPU-utilization).
These processes are then filtered into two lists, \textit{zombproc} for ``dead" processes and \textit{allproc} for ``live" ones.
This functionality is meant for cutting down the amount of work the \textit{wait} system call has to do, along with how many things the scheduler has to scan \cite{bsd:1}.

\subsubsection{Windows}
\label{sub:Process Windows}

\par Windows processes are also defined as programs in execution.
The standard data structure for a Windows process is called an EPROCESS.
Windows tracks running processes with the Process Executive Object Manager.
It does so by encapsulating the process in a ``process object" \cite{win:1}.
These structures are stored in system address space and are only accessible from kernel mode.
Almost all of the process creation takes place in kernel mode, which is Windows' way of preventing injection attacks. \cite{win:1}.
The only exception to that rule are programs that have been given debug privileges.
These programs are able to write arbitrary memory, inject code, resume threads, etc. \cite{win:1}

\par Other examples of process-relevant data structures in the Windows kernel are KPROCESS, CSR\_PROCESS, and W32PROCESS.
The most obvious of these being KPROCESS, which is a kernel process.
As the name implies, these are only accessible from kernel mode.
These are the only kind of processes running at the kernel level.
CSR\_PROCESS contains info specific to the Windows subsystem (Csrss - Client/Server Run-time Subsystem).
W32PROCESS is a very important kernel level process, because it tracks all pertinent information regarding the window management code and GUI processes.

\subsubsection{Compared to Linux}
\label{sub:Process Linux}

\par Both Windows and BSD view processes the same way as Linux.
Linux defines a process as a ``program (object code stored on some media) in the midst of execution" \cite{linux:1}.
In all three operating systems, processes exist in memory address space and are stored in their respective data structures.
Both FreeBSD and Linux use very similar data structures for storing process information.
They each have circular lists that store the PID of a process.
This differs from Windows, in that Windows uses a priority queue that it pops processes threads in and out of as their priorities change and their their turn arrives for CPU time.

\par Whereas Linux and FreeBSD differentiate processes in user and kernel space differently depending on the permission bestowed on execution, Windows handles this separation with entirely different data structures.
Windows has two different process data structures for user space and kernel space called EPROCESS (executive process) and KPROCESS (kernel process), respectively.
Another interesting thing about Windows is that it's kernel is built to help the OS as a work-station type operating system, rather than a headless one.
This is evident by analyzing the W32PROCESS, which is a process that stores information about window and GUI activity on the machine.
Linux and FreeBSD can both run without this functionality and only after introducing desktop software does it take advantage of its abstract data structures to make a user-friendly GUI a reality.

\subsection{Threads}
\label{sec:Threads}

\subsubsection{FreeBSD}
\label{sub:Thread FreeBSD}

\par In FreeBSD, a thread is referred to as a unit of execution of a process.
FreeBSD, similar to other distributions, implements the POSIX threading API commonly referred to as Pthreads.
If the operating system is installed on a multiprocessor, then it would permit multiple threads belonging to the same or different processes to execute concurrently.
CPU time is allocated to threads based on their priority, just like how processes are within the FreeBSD kernel.
This is coupled with their ``timeshare class", which is based on its resource usage and CPU usage \cite{bsd:1}.
Threads require an address space and other resources, but the resources provided can have shared access among other threads \cite{bsd:1}.

\par Threads are able to execute in either \textit{user mode} or \textit{kernel mode}.
User mode executes the program with a lower set of permissions and less access to the hardware and the kernel's internal functions.
Kernel mode is the opposite, where it is executed with heightened permissions and a potential shoot-yourself-in-the foot level of access to the hardware and kernel resources.
By default, threads within FreeBSD share all of the resources of their calling process, including the PID.

\subsubsection{Windows}
\label{sub:Thread Windows}

\par At a very high level, windows views threads very similarly to Unix.
From the reading, Windows describes a thread as ``[a] process that Windows schedules for execution" \cite{win:1}.
The data structure for a thread in Windows is referred to as an ETHREAD.
Like an EPROCESS (Windows process structure) all ETHREADS are stored in system address space.
A process can have many threads, but the initial ETHREAD is the primary executive thread for the process.

\par Threads in Windows have associated priorities that dictate their share of the CPU (or Quantum) by the scheduler.
We'll get into the Windows scheduler in the next subsection, but the interesting mechanic it implements to calculate a threads priority is called a boost.
A ``boost" is effectively a reason to give a thread priority based on the input it requires at that moment in execution.
There are a number of reason to boost a thread's priority, such as: scheduler events, I/O completion, UI input, waiting on a resource for too long, and ignored for too long.
Basically, this means that when something a thread is waiting on occurs or hasn't occurred for some time, that thread is promoted in priority to make the user experience more ergonomic.

\par Thread pools are a mechanism that Windows recently transitioned from being a user implemented tool to being completely managed by the kernel.
This change is due to the fact that the kernel allowed to directly manipulate thread creation, scheduling, and termination \cite{win:1}.
This allows the users program code to generate a worker pool through a simple API call, rather than managing virtual memory.

\subsubsection{Compared to Linux}
\label{sub:Process Linux}

\par Just like in FreeBSD, Linux allows the creation of threads within a process in the user space.
This differs from Windows, where the kernel handles all threads in worker pools and doles them out via API calls.
Multiprogramming and threads are handled almost the exact same way in FreeBSD and Linux.
Runnable threads are stored in "run queues" \cite{bsd:1} \cite{linux:1}, which is interacted with by calls to \textit{runq\_add()} and \textit{runq\_remove} based on priority of the process.
In linux, forks are almost identical to threads outside of a single flag passed to the \textit{clone()} function.
This extra flag allows resources to be shared, like the address space, file system resources, file descriptors, and signal handlers \cite{linux:1}.
FreeBSD tries to mimic this functionality with a call to \textit{pthread\_create()} when it wants a lightweight thread that shares resources.
Windows handles thread creation in the kernel with a call to \textit{PsCreateSystemThread} and a pointer to that thread is returned.

\par Another major difference between Linux and Windows, is how runnable threads are stored outside of their allotted CPU share.
Windows implements a priority queue that is maintained by the Windows scheduler.
Threads placed in the queue are done so after calculating their position based on a number of potentially relevant ``boosts".
These boosts differ from Linux in how they're several distinct factors for consideration when calculating priority, while Linux primarily focuses on an interactivity score.
FreeBSD and Linux share this approach by focusing on this interactivity score to provide a smooth user experience.
This is reflected in their data structure monitored by their scheduler, which is formatted as a linked list.

\subsection{CPU Scheduling}
\label{sec:CPU Scheduling}

\subsubsection{FreeBSD}
\label{sub:CPU Scheduling FreeBSD}

\par Processes require access to system resources like memory and CPU power.
The way the FreeBSD provides concurrent access to these resources to any number of running processes is through the use of a scheduler.
FreeBSD's default scheduler is called the timeshare scheduler.
The scheduler calculates a process' priority based on: CPU time used so far, amount of memory reserved for execution, and other factors.
\par In FreeBSD's timeshare scheduler exists a global list monitored by the scheduler that consists of runnable threads.
These threads and processes are the ``live" ones mentioned earlier in subsection \ref{sec:Processes}.
To prevent spending more time on context switching than actual process execution timeshares, the number of running processes must be capped at some point.
\par For usability purposes, the FreeBSD timeshare scheduler implements a policy that favors interactive programs over other pure-processor based processes \cite{bsd:1}.
Every thread has an ``interactivity score" that is calculated by the scheduler and results in a number from 0 to 100.
The interactivity threshold that is compared against for a thread or processes interactivity rating is calculated through ``nice" values to make sure batch processes still get to run.
This means that the cpu will attempt to handle any computation requested user interaction to ensure a smooth user experience.
\par Scheduling in FreeBSD is broken down into two parts: a simpler low-level scheduler that runs very often and a more powerful high-level scheduler that runs only a couple times per second \cite{bsd:1}.
The low-level scheduler handles whenever processes or threads block and another needs to be ran.
To simplify the decision making process for which goes when, the kernel keeps track of a run queue for each CPU sorted by priority.

\subsubsection{Windows}
\label{sub:CPU Scheduling Windows}

\par Process scheduling in Windows typically occurs at the thread level for desktop workstations, referred to in subsection \ref{sub:Thread Windows}.
Multi-user multi-processor is more for cluster environments, with batch computing and shared access.
This done through priority calculations based on the current need of a thread or process.
Scheduling at the thread-level is acceptable for legacy and single user machines, but for systems that cater to several competing users and utilize many CPUs it's not the answer.
Windows system for distributed fair share scheduling between multiple users is done through the session-based Distributed Fair Share Scheduler (DFSS).
A standard DFSS initialization set's the session's weight to 5, but this value can be anywhere between 1 to 9 \cite{win:1}.
This is what weight limits and throttles thread activity per session.

\par revisiting the more typical thread scheduling, one can see that Windows system for scheduling them is quite intuitive.
Threads that are set to be executed are queued up by priority.
The ready threads in the queue are stored in a data structure called the Dispatcher Database.
When it's time to pop a thread out of the queue, it's set to a ``running" state and stays that way for its reserved quantum of execution.
When the its share of CPU is up, then it will go back into the queue.
Since this queue is based on priority, it's important that no threads get starved and this is accomplished via the ``boosting" technique covered in subsection \ref{sub:Thread Windows}.

\subsubsection{Compared to Linux}
\label{sub:Sub Processes Linux}
\par The Linux scheduler is called the \textit{Completely Fair Scheduler}.
Policy is generally dictated outside of the scheduling technology as an after-thought, but the CFS was obviously built with it in mind.
The goal of this scheduler is to evenly distribute compute time between different processes and threads so that everyone got their fair share \cite{linux:1}.
The primary exception to this rule is an ``interactivity score", which gives priority to programs with high user interaction to provide a smooth user experience.
It also takes into account a process' given \textit{nice} score.
A high nice value says you're willing to surrender cpu time to other processes that need it more, whereas a lower one tells the processor to give you any extra.

\par The Linux scheduler differs from Windows in several ways and FreeBSD's ULE scheduler in less ways (a noticeable pattern).
One of the things these three operating systems schedulers have in common is they're all preemptive.
This means the highest priority process will always run.
Some of the ways the Windows scheduler differs is in it's usage of boosts (covered in subsection \ref{sub:Thread Windows}), the Dispatcher Database data structure, and it's different thread states.
The dispatcher database is a data structure that tracks which threads are ready and waiting to execute.
Windows allows for the possibility to have one Dispatcher Database per processor to improve scalability.

\par The Linux kernel places process in 3 different potential states. TASK\_RUNNING, TASK\_INTERRUPTIBLE, or TASK\_UNINTERRUPTIBLE.
This differs with the Windows Scheduler, in that they've many more states: Init, Ready, Running, Standby, Terminate, Waiting, Transition, and Deferred Ready \cite{win:1}.
The least obvious of these are the Transition and Deferred Ready.
Transition state means the thread is ready for execution, but the kernel stack is paged out of memory.
Once the kernel stack is back in memory, it will go back to being Ready.
Deferred Ready means this thread was scheduled for a particular processor, but it's currently busy.
The benefit to this structure is definitely the more descriptive labels for statuses, but it is definitely less abstract than the Linux/FreeBSD implementation.
