\documentclass[10pt,draftclsnofoot,onecolumn]{IEEEtran}

\usepackage{setspace}
% \usepackage{xcolor}
% \usepackage{caption}
\usepackage{listings}

\ifCLASSINFOpdf
  \usepackage[pdftex]{graphicx}
  \graphicspath{static/}
  \DeclareGraphicsExtensions{.pdf,.jpeg,.png}
\fi

% correct bad hyphenation here
\hyphenation{op-tical net-works semi-conduc-tor}

% listings options
\lstset{frame=lrbt,xleftmargin=\fboxsep,xrightmargin=-\fboxsep}

\begin{document}

\pagenumbering{gobble}
\singlespacing
\title{Project 3}

\author{Ty~Skelton}

% The paper headers
\markboth{CS 444}%
{Spring 2016}

% make the title area
\maketitle

% As a general rule, do not put math, special symbols or citations
% in the abstract or keywords.
\begin{abstract}
In this assignment I will describe my approach on how I introduced a new ram disk driver with added encryption.
This driver will be introduced to the kernel as a module, rather than being installed via config pre-launch.

\end{abstract}
\IEEEpeerreviewmaketitle

\newpage
\pagenumbering{arabic}

\tableofcontents
\newpage


\section{I/O Elevators}
\subsection{Main Point of Assignment}
The main point of this assignment, I believe, was actually two things.
Firstly, this assignment tests our ability to handle complex and strange libraries, specifically within the linux kernel.
The crypto library can be poorly documented and the only way to really use it is via other examples.
This is a necessary skill out in the field, because not always will there be documentation, let alone \textit{correct} documentation, but there will always be a lot of code floating around.
Lastly, how to develop and install a module.
This is a very useful skill, because not always will you want to embed the code in the kernel prior to running it.
Sometimes loading the module and removing it on the fly is crucial.

\subsection{Approach}
My approach stemmed from Kevin's original suggestion to check out http://lwn.net/Kernel/LDD3/ and search for ``sbd".
This led me to a bare-bones ram disk driver with just enough going on to allow me to seamlessly introduce my crypto code.
The crypto code itself was very easy to implement, since it was only a few lines.
The trickiest part to the entire assignment was figuring out how to load and install the module into the vm along with parameters.
I was able to find some cool linux documentation online with an example of someone using \texttt{insmod} command and was able to successfully cannibalize that code.

\subsection{Ensuring solution was correct}
I was able to ensure my solution was correct very simply by printing out to the terminal.
This proved I was actually running my code and coupled with printing out my encrypted text proved it was working.
This is what I usually do when trying to prove that it works, because I like that as it's running I can print out the values and see.

\subsection{What was learned}
I learned that to get things done sometimes you have to scrape for quality examples.
I also learned that linux doesn't have the most friendly documentation.
There's a lot of life lessons that can be learned from the linux kernel.
It's that the world is full of liars and that you shouldn't trust anyone.
Most importantly, I learned how to install a module and that's pretty cool.

\section{Set Up}
Building module and loading it into qemu
\begin{list}{-}{}
\item Makefile next to your sbd.c code \\
\texttt{obj-m := sbd.o}
\item run
\texttt{make -C ../linux-yocto-3.14 SUBDIRS=\$PWD modules}
\item get module onto server \\
\texttt{scp skeltont@os-class.engr.oregonstate.edu:~/modules/sbd.ko ./}
\item install module to server \\
\texttt{insmod sbd.ko crypto\_key="testtest11"}
\item choose the driver \\
\texttt{mkfs.ext2 /dev/tysbd0}
\item make the testdir and mount it \\
\texttt{mkdir /mnt/testdir}
\texttt{mount /dev/tysbd0 /mnt/testdir}
\item voila
\end{list}

\section{Version Control Log}
\begin{tabular}{l l l}\textbf{Commit} & \textbf{Author} & \textbf{Description}\\\hline
\ d871807 & Ty Skelton & initial commit\\\hline
\ 58fcc8c & Ty Skelton & clean master branch to build from for different devices\\\hline
\ fd268d2 & Ty Skelton & cleans up repository structure and files\\\hline
\ 4d9d85e & Ty Skelton & adds config\\\hline
\ 670cd9a & Ty Skelton & reorganizes file structure:\\\hline
\ f45c406 & Ty Skelton & Merge pull request 1 from skeltont/os-class\\\hline
\ 54d3081 & Ty Skelton & adds first half of assignment to write-up\\\hline
\ c6f7bee & Ty Skelton & adds report, concurrency assignment1\\\hline
\ a15697e & Ty Skelton & fixes rdrand check failing on os-class and adds makefile for concurrency\\\hline
\ 9cd4722 & Ty Skelton & adds output style to concurrency, fixes make file for writeup\\\hline
\ 10ecc5f & Ty Skelton & fixes rdrand capability checking\\\hline\end{tabular}


\section{Work Log}
\begin{tabular}{l l}\textbf{Date} & \textbf{Description}\\\hline
\ May 28 & Duplicated clean kernel and added necessary config files and run scripts.\\\hline
\ May 29 & Successfully loaded slob memory manager into kernel.\\\hline
\ June 1 & Non-working changes for trying to implement best fit.\\\hline
\ June 2 & Break-through with getting best fit to work. simple code and small change to have new page on first call.\\\hline
\ June 4 & Wrote the report for the assignment and submit homework\\\hline
\end{tabular}


\end{document}
