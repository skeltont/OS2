\documentclass[10pt,draftclsnofoot,onecolumn]{IEEEtran}

\usepackage{setspace}
% \usepackage{xcolor}
% \usepackage{caption}
\usepackage{listings}

\ifCLASSINFOpdf
  \usepackage[pdftex]{graphicx}
  \graphicspath{static/}
  \DeclareGraphicsExtensions{.pdf,.jpeg,.png}
\fi

% correct bad hyphenation here
\hyphenation{op-tical net-works semi-conduc-tor}

% listings options
\lstset{frame=lrbt,xleftmargin=\fboxsep,xrightmargin=-\fboxsep}

\begin{document}

\pagenumbering{gobble}
\singlespacing
\title{Project 3}

\author{Ty~Skelton}

% The paper headers
\markboth{CS 444}%
{Spring 2016}

% make the title area
\maketitle

% As a general rule, do not put math, special symbols or citations
% in the abstract or keywords.
\begin{abstract}
In this assignment I will describe how I was able to implement best fit into the slob memory manager.
I will provide answer to several questions about my process and what I learned, along with logs for work and version control.
This report is accompanied by a patch file that reflects my changes.
\end{abstract}
\IEEEpeerreviewmaketitle

\newpage
\pagenumbering{arabic}

\tableofcontents
\newpage


\section{I/O Elevators}
\subsection{Main Point of Assignment}
The main point of this assignment I believe was to teach us about how memory managers work and to see a \textit{really bad one} in action.
I had a faint understanding of how pages were implemented and managed when I was doing the assigned readings, but after getting into the slob manager it really demystified a lot for me.
In reference to my ``really bad one" statement earlier, when using the slob manager I let it boot on it's own without any changes and it just never started.
So I therefore learned to not take for granted the things modern operating systems are able to pull off.

\subsection{Approach}
My approach for this assignment was the same as all of the others so far.
First I made to sure I was able to successfully get into the boot stage with the slob memory manager.
After that I started to make print statements to see what kind of data I was working with.
I talked to some other students and got a feel for some problem areas and then started to make meaningful changes.
Early on I had a lot of issues with getting it to actually work, but I figured out I didn't have a base case for allocating a page when there were none.
Fixing that meant I was done and it was exciting.

\subsection{Ensuring solution was correct}
To ensure the solution was correct I used plenty of print statements to tell me through the console when something became a best fit, was default, etc.

\subsection{What was learned}
I learned that modern operating systems are really well-tuned machines for having such good memory management systems.
The slob and my variation were both very slow and it seems tough to do this well.
I learned it wasn't as hard as I thought it was gonna be, but that doesn't mean much because it was still hard.
Overall I'd say (since this is the last write-up for the class) I learned a lot about kernel development as a whole and have found a higher appreciation for it.
I do not think I'll ever find myself in this field of work, but I can definitely say I've considered it now.

\section{Version Control Log}
\begin{tabular}{l l l}\textbf{Commit} & \textbf{Author} & \textbf{Description}\\\hline
\ d871807 & Ty Skelton & initial commit\\\hline
\ 58fcc8c & Ty Skelton & clean master branch to build from for different devices\\\hline
\ fd268d2 & Ty Skelton & cleans up repository structure and files\\\hline
\ 4d9d85e & Ty Skelton & adds config\\\hline
\ 670cd9a & Ty Skelton & reorganizes file structure:\\\hline
\ f45c406 & Ty Skelton & Merge pull request 1 from skeltont/os-class\\\hline
\ 54d3081 & Ty Skelton & adds first half of assignment to write-up\\\hline
\ c6f7bee & Ty Skelton & adds report, concurrency assignment1\\\hline
\ a15697e & Ty Skelton & fixes rdrand check failing on os-class and adds makefile for concurrency\\\hline
\ 9cd4722 & Ty Skelton & adds output style to concurrency, fixes make file for writeup\\\hline
\ 10ecc5f & Ty Skelton & fixes rdrand capability checking\\\hline\end{tabular}


\section{Work Log}
\begin{tabular}{l l}\textbf{Date} & \textbf{Description}\\\hline
\ May 28 & Duplicated clean kernel and added necessary config files and run scripts.\\\hline
\ May 29 & Successfully loaded slob memory manager into kernel.\\\hline
\ June 1 & Non-working changes for trying to implement best fit.\\\hline
\ June 2 & Break-through with getting best fit to work. simple code and small change to have new page on first call.\\\hline
\ June 4 & Wrote the report for the assignment and submit homework\\\hline
\end{tabular}


\end{document}
