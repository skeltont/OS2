\documentclass[10pt,draftclsnofoot,onecolumn]{IEEEtran}

\usepackage{setspace}
% \usepackage{xcolor}
% \usepackage{caption}
\usepackage{listings}

\ifCLASSINFOpdf
  \usepackage[pdftex]{graphicx}
  \graphicspath{static/}
  \DeclareGraphicsExtensions{.pdf,.jpeg,.png}
\fi

% correct bad hyphenation here
\hyphenation{op-tical net-works semi-conduc-tor}

% listings options
\lstset{frame=lrbt,xleftmargin=\fboxsep,xrightmargin=-\fboxsep}


\begin{document}

\pagenumbering{gobble}
\singlespacing
\title{Project 1}

\author{Ty~Skelton}


% The paper headers
\markboth{CS 444}%
{Spring 2016}

% make the title area
\maketitle

% As a general rule, do not put math, special symbols or citations
% in the abstract or keywords.
\begin{abstract}
The first project of Operating Systems 2.
This assignment is meant to introduce us to working with the the linux kernel and writing concurrent programs in C.


\end{abstract}
\IEEEpeerreviewmaketitle

\newpage
\pagenumbering{arabic}

\section{Booting the Kernel on the VM}

\subsection{Log of Commands}

\subsubsection{}
Acquiring a local copy of the Kernel by running
\begin{lstlisting}[language=bash]
  $ git clone git://git.yoctoproject.org/linux-yocto-3.14
\end{lstlisting}

\subsubsection{}
Copying over all the necessary files into the root of my linux tree:
\begin{lstlisting}[language=bash]
  $ cp /scratch/spring2015/files/config-3.14.26-yocto-qemu ./.config
  $ cp /scratch/spring2015/files/bzImage-qemux86.bin ./
  $ cp /scratch/spring2015/files/core-image-lsb-sdk-qemux86.ext3 ./
\end{lstlisting}

\subsubsection{}
Building the kernel:
\begin{lstlisting}[language=bash]
  $ make -j4 all
\end{lstlisting}

\subsubsection{}
Writing a run script:
\begin{lstlisting}[language=bash]
  #!/bin/bash

  source /scratch/opt/environment-setup-i586-poky-linux

  qemu-system-i386 -gdb tcp::5618 -S -nographic -kernel bzImage-qemux86.bin \
  -drive file=core-image-lsb-sdk-qemux86.ext3,if=virtio \
  -enable-kvm -net none -usb -localtime --no-reboot \
  --append "root=/dev/vda rw console=ttyS0 debug"
\end{lstlisting}

\subsubsection{}
Running the script for the first time:
\begin{lstlisting}[language=bash]
  $ chmod u+x run
  $ ./run
\end{lstlisting}

\subsubsection{}
Creating the gdb initializer script:
\begin{lstlisting}[language=bash]
  target remote :5618
  symbol-file linux-yocto-3.14/vmlinux
\end{lstlisting}

\subsubsection{}
Connecting gdb from another shell:
\begin{lstlisting}[language=bash]
  $ gdb
\end{lstlisting}

\subsubsection{}
After typing continue in the gdb instance, I was able to successfully login with the credentials of root.

\subsection{Qemu CLI Flags}
\textit{-gdb tcp::5618}
This flag will tell Qemu to open a gdb server on the following device.
We specify to a reserved tcp port.

\textit{-S}
This flag instructs Qemu to not start the CPU at start up and to wait for a continue from the device monitor.

\textit{-nographic}
Normally Qemu displays output to VGA.
With this flag it will bypass that entirely and spin up a headless command line application.

\textit{-kernel bzImage-qemux86.bin}
Specifies the particular kernel to use.

\textit{-drive file=core-image-lsb-sdk-qemux86.ext3,if=virtio}
This flag specifies the drive to use, with some following options.
The file option defines a disk image and the if option defines the type of interface the device is connected to.

\textit{-enable-kvm}
This flag enables full KVM (Kernel-based Virtual Machine) support.

\textit{-net none}
Instructs the VM that no network devices should be configured.

\textit{-usb}
Enables the USB drivers.

\textit{-localtime}
Sets the time to the localtime of the calling machine.

\textit{--no-reboot}
Exits rather than rebooting.

\textit{--append "root=/dev/vda rw console=ttyS0 debug"}
Sends command line arguments to the kernel.

\end{document}
