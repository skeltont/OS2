\documentclass[10pt,draftclsnofoot,onecolumn]{IEEEtran}

\usepackage{setspace}
% \usepackage{xcolor}
% \usepackage{caption}
\usepackage{listings}

\ifCLASSINFOpdf
  \usepackage[pdftex]{graphicx}
  \graphicspath{static/}
  \DeclareGraphicsExtensions{.pdf,.jpeg,.png}
\fi

% correct bad hyphenation here
\hyphenation{op-tical net-works semi-conduc-tor}

% listings options
\lstset{frame=lrbt,xleftmargin=\fboxsep,xrightmargin=-\fboxsep}

\begin{document}

\pagenumbering{gobble}
\singlespacing
\title{Project 3}

\author{Ty~Skelton}

% The paper headers
\markboth{CS 444}%
{Spring 2016}

% make the title area
\maketitle

% As a general rule, do not put math, special symbols or citations
% in the abstract or keywords.
\begin{abstract}
In this assignment I will describe my approach on how I introduced a new ram disk driver with added encryption.
This driver will be introduced to the kernel as a module, rather than being installed via config pre-launch.

\end{abstract}
\IEEEpeerreviewmaketitle

\newpage
\pagenumbering{arabic}

\tableofcontents
\newpage


\section{I/O Elevators}
\subsection{Main Point of Assignment}
The main point of this assignment, I believe, was actually two things.
Firstly, this assignment tests our ability to handle complex and strange libraries, specifically within the linux kernel.
The crypto library can be poorly documented and the only way to really use it is via other examples.
This is a necessary skill out in the field, because not always will there be documentation, let alone \textit{correct} documentation, but there will always be a lot of code floating around.
Lastly, how to develop and install a module.
This is a very useful skill, because not always will you want to embed the code in the kernel prior to running it.
Sometimes loading the module and removing it on the fly is crucial.

\subsection{Approach}
My approach stemmed from Kevin's original suggestion to check out http://lwn.net/Kernel/LDD3/ and search for ``sbd".
This led me to a bare-bones ram disk driver with just enough going on to allow me to seamlessly introduce my crypto code.
The crypto code itself was very easy to implement, since it was only a few lines.
The trickiest part to the entire assignment was figuring out how to load and install the module into the vm along with parameters.
I was able to find some cool linux documentation online with an example of someone using \texttt{insmod} command and was able to successfully cannibalize that code.

\subsection{Ensuring solution was correct}
I was able to ensure my solution was correct very simply by printing out to the terminal.
This proved I was actually running my code and 


\subsection{What was learned}


\section{Set Up}
Building module and loading it into qemu
\begin{list}{-}{}
\item Makefile next to your sbd.c code \\
\texttt{obj-m := sbd.o}
\item run
\texttt{make -C ../linux-yocto-3.14 SUBDIRS=\$PWD modules}
\item get module onto server \\
\texttt{scp skeltont@os-class.engr.oregonstate.edu:~/modules/sbd.ko ./}
\item install module to server \\
\texttt{insmod sbd.ko crypto\_key="testtest11"}
\item choose the driver \\
\texttt{mkfs.ext2 /dev/tysbd0}
\item make the testdir and mount it \\
\texttt{mkdir /mnt/testdir}
\texttt{mount /dev/tysbd0 /mnt/testdir}
\item voila
\end{list}

\section{Version Control Log}
\begin{tabular}{l l l}\textbf{Commit} & \textbf{Author} & \textbf{Description}\\\hline
\ 2f92a8f & Ty Skelton & adds config setting that selects slob as the default memory manager.\\\hline
\ 9a47b6f & Ty Skelton & adds logging to show that slob is working and what datatypes relying on.\\\hline
\ 6b33426 & Ty Skelton & adds non-working code that is a step in the right direction i think.\\\hline
\ 6b33426 & Ty Skelton & changes non-working code and adds default case that was breaking it.\\\hline
\end{tabular}


\section{Work Log}
\begin{tabular}{l l}\textbf{Date} & \textbf{Description}\\\hline
\ March 29 & Created private repository on GitHub and began setting up environment for development.\\\hline
\ March 31 & Successfully built kernel and attached it to GDB after creating some cool config files.\\\hline
\ April 4  & Created LaTex document and wrote first half of write-up regarding the kernel build (A \& B). \\\hline
\ April 5  & Started working on concurrency assignment and was able to achieve base functionality. \\\hline
\ April 7  & Finished concurrency assignment, remaining work was to do in-line assembly. Added color flair.\\\hline
\ April 8  & Finished writing up concurrency assignment and verified all make files worked on os-class. \\\hline
\end{tabular}


\end{document}
