\documentclass[10pt,draftclsnofoot,onecolumn]{IEEEtran}

\usepackage{setspace}
\usepackage{listings}

% correct bad hyphenation here
\hyphenation{op-tical net-works semi-conduc-tor}

\begin{document}

\pagenumbering{gobble}
\singlespacing
\title{Weekly Summary 4}

\author{Ty~Skelton}

% The paper headers
\markboth{CS 444}
{Spring 2016}

% make the title area
\maketitle
\IEEEpeerreviewmaketitle

% 1) In a single coherent sentence give the following:
%   i)   name of the author, title of the work, date in parenthesis;
%   ii)  a rhetorically accurate verb (such as "assert," "argue," "deny," "refute," "prove," "disprove," "explain," etc.);
%   iii) a that clause containing the major claim (thesis statement) of the work.
Robert Love, "Linux Kernel Development" (June 2010) chapters 6 \& 7, describes the data structures within the kernel and how it handles interrupts.
% 2) In a single coherent sentence give an explanation of how the author develops and supports the major claim (thesis statement).
Love is able to do this by detailing core data structures (e.g. Linked Lists, Queues, etc.) used throughout the kernel and by providing a practical example of an interrupt in the RTC libraries.
% 3) In a single coherent sentence give a statement of "The author's purpose is _____", followed by an "in order to ______" phrase.
The author's purpose is to familiarize the reader with data structures and handling interrupts, in order to prevent them from creating their own data structures and using the right ones for interrupts.  
% 4) In a single coherent sentence give a description of the intended audience and/or the relationship the author establishes with the audience.
The intended audience for this work has been explicitly described as Linux developers and users who are interested in understanding the Linux Kernel.
\end{document}
