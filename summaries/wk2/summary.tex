\documentclass[10pt,draftclsnofoot,onecolumn]{IEEEtran}

\usepackage{setspace}
% \usepackage{xcolor}
% \usepackage{caption}
\usepackage{listings}

\ifCLASSINFOpdf
  \usepackage[pdftex]{graphicx}
  \graphicspath{static/}
  \DeclareGraphicsExtensions{.pdf,.jpeg,.png}
\fi

% correct bad hyphenation here
\hyphenation{op-tical net-works semi-conduc-tor}

% listings options
\lstset{frame=lrbt,xleftmargin=\fboxsep,xrightmargin=-\fboxsep}


\begin{document}

\pagenumbering{gobble}
\singlespacing
\title{Weekly Summary 2}

\author{Ty~Skelton}


% The paper headers
\markboth{CS 444}%
{Spring 2016}

% make the title area
\maketitle
\IEEEpeerreviewmaketitle

% 1) In a single coherent sentence give the following:
%   i)   name of the author, title of the work, date in parenthesis;
%   ii)  a rhetorically accurate verb (such as "assert," "argue," "deny," "refute," "prove," "disprove," "explain," etc.);
%   iii) a that clause containing the major claim (thesis statement) of the work.
Robert Love, "Linux Kernel Development" (June 2010), defines how linux supports processes and the scheduler that handles them.
% 2) In a single coherent sentence give an explanation of how the author develops and supports the major claim (thesis statement).
Love is able to provide insight on this topic by going in-depth on both the user/system-level calls that make processor and memory virtualization possible in a concurrent manner.
% 3) In a single coherent sentence give a statement of the author's purpose, followed by an "in order" phrase.
The author provides great detail when exploring these topics with the reader in order to establish a base understanding of how the Linux kernel views processes resource shares, address spaces, and other fundamentals before we continue on learning more advanced topics.
% 4) In a single coherent sentence give a description of the intended audience and/or the relationship the author establishes with the audience.
The intended audience for this work has been explicitly described as Linux developers and users who are interested in understanding the Linux Kernel, whom which Love approaches in a teacher/student context.
\end{document}
